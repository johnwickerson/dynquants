\documentclass[a4paper]{llncs}

% Page size (with wide righthand margin, for todo notes)
\usepackage[paperwidth=21cm, paperheight=22cm, textwidth=12.2cm,
textheight=19.3cm]{geometry}
\addtolength\hoffset{-3cm}
\addtolength\marginparwidth{3cm}

% Use Times font (it's more compact)
%\renewcommand\rmdefault{ptm}

% Define a version of \paragraph whose heading runs straight into the
% text that follows.
% {
\makeatletter
\usepackage{suffix}
\renewcommand\paragraph[1]{%
\@startsection{paragraph}{3}{\z@}%
{-12\p@ \@plus -4\p@ \@minus -4\p@}%
{-0.5em \@plus -0.22em \@minus -0.1em}%
{\normalfont\normalsize\bfseries\boldmath}{#1}}
\WithSuffix\newcommand\paragraph*[1]{%
\@startsection{paragraph}{3}{\z@}%
{-12\p@ \@plus -4\p@ \@minus -4\p@}%
{0pt}%
{\normalfont\normalsize\bfseries\boldmath}{#1} \hspace*{0pt}}
\makeatother
%}

% Line numbers in margin
\usepackage{lineno}
\setpagewiselinenumbers
\modulolinenumbers[5]

% Reduce spacing above/below figures
%\setlength\textfloatsep{10.0pt plus 2.0pt minus 4.0pt}
%\setlength\floatsep{6.0pt plus 2.0pt minus 2.0pt}


\usepackage{ifthen}
\usepackage{array}
\usepackage[T1]{fontenc}
\usepackage{url}


% Inline lists
\usepackage[inline]{enumitem}
\newenvironment{inlineitemize}{
\begin{itemize*}[label={},afterlabel={},before=\unskip]
}{
\end{itemize*}
}

% Remove spacing above/below lists
%\setlist{noitemsep, topsep=0pt} 

% Prettier tables
\usepackage{booktabs}

% Multiple footnotes per line
\usepackage[para]{footmisc}

%\usepackage[colorlinks=true,allcolors=blue,breaklinks,draft=false]{hyperref}

\usepackage{amsmath}
%\usepackage{amsthm}
\usepackage{amssymb}
\usepackage{stmaryrd}
\usepackage[usenames,dvipsnames]{xcolor}

% Comments
\usepackage{todonotes}
\setcounter{tocdepth}{1} % Makes \listoftodos work with llncs
\newcommand\JWComment[1]{\todo[backgroundcolor=blue!10,
  linecolor=blue]{{\bf John:} #1}}

% Code listings
\usepackage{listings}
\lstset{
  basicstyle=\small\ttfamily,
  basewidth=0.5em, % when using cmtt
  columns=fixed,
  mathescape=true,
  numberstyle=\scriptsize,
  escapechar=",
  xleftmargin=5mm,
  numbersep=3mm,
}

\usepackage{tikz}
\usetikzlibrary{matrix,fit,calc,positioning}
\tikzset{event/.style={draw=none, inner sep=0.4mm}}

% For adding borders/shading to blocks of text
\usepackage{framed}
\definecolor{shadecolor}{gray}{0.9}

% For highlighting text
\usepackage{soul} 
\definecolor{highlightingcolor}{HTML}{FFFF99}
\sethlcolor{highlightingcolor}
\newcommand\mhl[1]{\fboxsep=0pt\colorbox{highlightingcolor}{\strut$#1$}}

\newcommand\stack[2][l]{{\renewcommand\arraystretch{1}\begin{array}[t]{@{}#1@{}}#2\end{array}}}

\newcommand\var[1]{\mathit{#1}}
\newcommand\code[1]{\mbox{\lstinline{#1}}}
\newcommand\sem[1]{\llbracket #1 \rrbracket}
\newcommand\roundsem[1]{\llparenthesis\hspace{0.1ex}#1\hspace{0.1ex}\rrparenthesis}
\newcommand\semicolon{\mathbin{;}}

\title{Dynamic quantifiers and proof graphs}
\author{John Wickerson}
\institute{Imperial College London}

\begin{document}

\maketitle

%\linenumbers

\pagestyle{plain}

\begin{abstract} 
To do.
\end{abstract}

\section{Introduction}

Let us consider the following fragment of first-order logic:
\begin{align*}
P &::= f(\bar x) \mid P \wedge P \mid P \vee P \mid \exists x \ldotp P
\end{align*}
where $f$ ranges over a set $\mathbb F$ of function symbols and $x$
ranges over a set $\mathbb X$ of variables. (Since it admits no
negative formulae, our fragment is rather impoverished, but it is
roughly as expressive as the language of `symbolic heaps' that is
heavily used in the study of separation logic.) If $<$ is
a 2-ary function symbol in $\mathbb F$, one formula in our
language is:
\begin{equation}
(\exists z\ldotp \exists y\ldotp 0 < y \wedge y < z \wedge z < w)
\wedge w < 4. \label{eq:formula1}
\end{equation}

We would like to normalise formulae in our language. (Normalisation is
useful because two formulae sharing the same normal form can be deemed
equivalent.) The standard way to do this involves pulling the
quantifiers as far `outside' as possible (exploiting the identity
$P\wedge (\exists x\ldotp Q) \equiv \exists x \ldotp P \wedge Q$ when
$x$ is not free in $P$) and then introducing the quantifiers in a
canonical order, say alphabetical (exploiting the identity $\exists y
\ldotp \exists x\ldotp P \equiv \exists x\ldotp\exists y \ldotp
P$). Applying this process to \eqref{eq:formula1} yields:
\begin{equation}
\exists y\ldotp \exists z\ldotp 0 < y \wedge y < z \wedge z < w
\wedge w < 4. \label{eq:formula2}
\end{equation}

One disadvantage of pulling quantifiers to the `outside' of a formula
is the lack of congruity. For instance, if $P \Rightarrow Q$, then
even if $C[P]$, $P$ and $Q$ are all in normal form, $C[Q]$ need not be.








\bibliographystyle{plain}
\bibliography{dynquants}

%\pagebreak

%\listoftodos


\end{document}

%%% Local Variables:
%%% mode: latex
%%% TeX-master: t
%%% End:
